%===================================================PREÂMBULO==========================================================

\documentclass[12pt,a4paper]{article}
\usepackage[utf8]{inputenc}
\usepackage[brazil]{babel}
\usepackage[T1]{fontenc}
\usepackage{amsmath}
\usepackage{amsfonts}
\usepackage{amssymb}
\usepackage{graphicx}
\usepackage[left=2cm,right=2cm,top=2cm,bottom=2cm]{geometry}
\author{Leandro Custódio}
\title{Relatório do Programa de Monitoria Acadêmica - Promac}
\usepackage{indentfirst}
\usepackage{setspace}
\usepackage{fancyhdr}
%\usepackage[fontsize=12pt]{scrextend}

%==================================================DOCUMENTO===========================================================
\begin{document}

	\maketitle  %Este é apenas uma comentário para testar o commit
	
	\pagebreak
	
    \tableofcontents \newpage

    \listoffigures \newpage

    \listoftables \newpage

    \section{Introdução}
    
    	\onehalfspacing
    	O Programa de Monitoria Acadêmica - PROMAC tem como objetivo incentivar a participação dos alunos dos cursos de 				graduação nas atividades de iniciação à docência universitária, proporcionando visão integrada e contextualizada da 			disciplina/área em que o monitor realiza as atividades acadêmicas. É mantido por meio de recursos do Fundo de Combate 		à Pobreza - FECOP e de custeio da própria UECE. Além disso, oferece um grande número de oportunidades através de 				vagas para a monitoria voluntária.

    	O presente relatório tem o objetivo de fazer uma pequena exposição dos dados referentes ao PROMAC no período de 2015 			à 2019. Este período foi escolhido com base nos dados coletados no sistema de gerenciamento de bolsas da UECE, Bolsas 		- UECE.
    	
    \section{Metodologia Empregada}
    
    	Com o intuito de obter uma melhor compreensão de como o PROMAC tem se desenvolvido no transcorrer da janela temporal 			escolhida para a geração deste relatório, empregamos uma metodologia que acreditamos ser a mais adequada para este 				caso.

    	Foram elaboradas algumas perguntas que nortearam o nosso olhar no processo de análise descritiva do dados. Por sua 				vez, como base nestas perguntas, buscou-se um melhor entendimento, embora de forma incipiente, do que tem 						transcorrido nos últimos anos. Em seguida, apresentamos a lista de perguntas mencionada.
    	
    	\subsection{Lista de Perguntas}
        Nesta seção, apresentamos as perguntas utilizadas para nos orientar durante a manipulação dos dados.

        	\subsubsection{Quantos alunos passaram pelo PROMAC durante o período estudado?}

        	\subsubsection{Quantos alunos passaram pelo PROMAC por ano?}

        	\subsubsection{Quantos Alunos Passaram pelo PROMAC Por Curso?}

       		\subsubsection{Quantos alunos Passaram pelo PROMAC por Centro/Faculdade?}

        	\subsubsection{Quanto foi investido no PROMAC?}

        	\subsubsection{Qual a distribuição dos monitores por fonte de pagamento?}

    \pagebreak

    \section{Análise dos Dados}
    Nesta Seção, nos empenhamos em responder as perguntas realizadas na lista anteriormente proposta com base nos dados 			obtidos.

        \subsection{Quantos alunos passaram pelo PROMAC durante o período estudado?}
        Com base em dados extraídos do sistema bolsas, vemos que, durante o período analisado, anos de 2015 - 2020, passaram 			pelo PROMAC um total de 2.140 alunos. Vale salientar que o referido período representa apenas os alunos que foram 				registrados no sistema, que foi implementado em 2015.

        Se levarmos em consideração que, segundo dados obtidos Através do Google, a UECE possuía, em 2013, um total de 21.653 		alunos, e que a cada ano ingressam, em média, 2117 alunos, e colam grau, em média, 1193, podemos estimar que a UECE 			tem atualmente algo em torno de  28.811 alunos.

        Sendo assim, quantidade de alunos que passou pelo PROMAC em relação a população atual, representa algo em torno de 				7,42 \%. Para que fique mais claro, expomos a representação gráfica da informação acima mencionada.
        
        \begin{figure}[htb]
        	\centering
        	\includegraphics[scale=0.6]{Ingressantes.Ano.png}
        	\caption{Ingressantes por Ano}
        	\label{Ingressantes-Ano}
    	\end{figure}
    	
    	\pagebreak
    	
    	Podemos ver tambem os dados em forma tabular. \\

    	\begin{table}[!htb]
        	\begin{center}
            	\begin{tabular}{|c|c|}
                	\hline
                    Ano & Quantidade de Alunos \\ \hline
                    2014 & 2.281 \\ \hline
                    2015 & 1.754 \\ \hline
                    2016 & 1.821 \\ \hline
                    2017 & 1.583 \\ \hline
                    2018 & 1.734 \\ \hline
                    2019 & 1.914 \\ 
                	\hline
            	\end{tabular}
            	\caption{Quantidade de Alunos Ingressantes por Ano.}
            	\label{Tabela1}
        	\end{center}
    	\end{table}
    	
    	%https://www.google.com/search?q=quantos+alunos+tem+a+UECE&oq=quantos+alunos+tem+a+UECE&aqs=chrome..							69i57.10126j1j7&sourceid=chrome&ie=UTF-8

		%2014.1 = 2,281
		%https://diariodonordeste.verdesmares.com.br/metro/uece-divulga-resultado-da-1-fase-do-vestibular-2014-1-1.856619

		%2015.2 = 1,754
		%https://www.ceara.gov.br/2015/04/09/uece-oferece-1754-vagas-para-o-vestibular-20152/

		%2016.2 = 1,821
		%http://g1.globo.com/ceara/noticia/2016/04/universidade-estadual-do-ceara-abre-inscricao-do-vestibular-201.html

		%2017.1 = 1,583
		%https://vestibular.brasilescola.uol.com.br/noticias/uece-publica-resultado-vestibular-2017-1/338325.html

		%2018 = 1734
		%https://vestibular.brasilescola.uol.com.br/noticias/resultado-1-a-fase-vestibular-2018-2-uece-esta-disponivel/					343225.html#:~:text=A%20UECE%20oferece%201.734%20vagas,)%203101%2D9710%20%2F%209711.

		%2019 = 1,914
		%https://vestibular.brasilescola.uol.com.br/noticias/uece-publica-aprovados-no-vestibular-2019-2/346088.html



		\subsection{Quantos alunos passaram pelo PROMAC por ano?}
		
		Nesta seção, apresentamos informações sobre o números de monitores que participaram do PROMAC durante cada ano no 				intervalo iniciado em 2015 até o ano de 2019. Segundo o gráfico abaixo, percebemos que a demanda dos alunos pela 				participação no Programa, tem aumentado ao longo dos anos. Parte desse fenômeno, se dar pelo fato de que o programa 			em si teve o seu número de vagas ofertadas aumentado.
		
		Cabe aqui salientar que os números apresentados, algumas vezes são maiores que os números de vagas dos seus 					respectivos anos. Para evitar dúvidas, fazemos questão de informar que estes números referem-se à quantidade 					absolutas de alunos que passaram pelo programa, ou seja, foi levado em consideração a rotatividade de alunos no 				programa. Muitas vezes, essa rotatividade se dá, pois alguns alunos colam grau, mudam para bolsas de assistência, 				pesquisa, extensão. Outras vezes, os alunos ingressam no mercado de trabalho. 
		
			\begin{figure}[htb]
        		\centering
        		\includegraphics[scale=0.6]{Figure_2.png}
        		\caption{Monitores por Ano}
        		\label{Monitores-Ano}
    		\end{figure}
    		
    	\pagebreak
    	
    	Com intuito de deixar as informações mais claras, construímos a seguinte tabela.

			\begin{table}[!htb]
        	\begin{center}
            	\begin{tabular}{|c|c|}
                	\hline
                    Ano & Quantidade de Monitores por Ano\\ \hline
                    2015 & 302 \\ \hline
                    2016 & 252 \\ \hline
                    2017 & 471 \\ \hline
                    2018 & 495 \\ \hline
                    2019 & 600 \\ 
                	\hline
            	\end{tabular}
            	\caption{Quantidade de Monitores por Ano.}
            	\label{Tabela2}
        	\end{center}
    		\end{table}
    		
    	
		\subsection{Quantos Alunos Passaram pelo PROMAC Por Curso?}
		
		Nesta seção, fizemos o levantamento do numero de monitores que cada curso da Instituição teve no referido período com 		o intuito de verificar quais cursos são os mais atuantes no programa. Para isso, cosultemos a tabela seguinte:
		
			\begin{table}[!htb]
        	\begin{center}
            	\begin{tabular}{|p{4cm}|c|p{4cm}|c|}
                	\hline
                    Curso & QTD. de Alunos & Curso & QTD. de Alunos\\ \hline
                    ADMINISTRACAO & 68 &  LETRAS & 150 \\ \hline
                    ARTES VISUAIS & 5 & MATEMATICA & 102 \\ \hline
                    CIENCIA DA COMPUTACAO & 23 & MEDICINA & 200 \\ \hline
                    CIENCIAS BIOLOGICAS & 364 & MEDICINA VETERINARIA & 203 \\ \hline
                    CIENCIAS CONTABEIS & 34 & MUSICA & 5 \\ \hline
                    CIENCIAS SOCIAIS & 27 & NUTRICAO & 193 \\ \hline
                    EDUCACAO FISICA & 80 & PEDAGOGIA & 258 \\ \hline
                    ENFERMAGEM & 157 & PSICOLOGIA & 44 \\ \hline
                    FILOSOFIA & 109 & SERVICO SOCIAL & 68 \\ \hline 
                    FISICA & 105 & QUIMICA & 298 \\ \hline
                    HISTORIA & 126 & SERVICO SOCIAL &68 \\ 
                    \hline
            	\end{tabular}
            	\caption{Quantidade de Monitores por Curso.}
            	\label{Tabela3}
        	\end{center}
    		\end{table}
    		
    	\pagebreak

		\subsection{Quantos alunos Passaram pelo PROMAC por Centro / Faculdade?}
		
		Nesta seção, verificamos quantos monitores, por centro/faculdades participaram do programa. Vejamos as informações no 		gráfico seguinte:
			
			\begin{figure}[htb]
        		\centering
        		\includegraphics[scale=0.6]{Figura_3.png}
        		\caption{Monitores por Centro/Faculdade}
        		\label{Monitores-Centro/Faculdade}
    		\end{figure}
    		

		\subsection{Quanto foi investido no PROMAC?}
		
		A UECE, ao longo do período analisado, investiu ao todo a quantia de R\$ 1.756.150,00(Um Milhão, Setecentos e 					Cinquenta e Seis mil, Cento e Cinquenta reais) com o intuito de manter o programa. Desse valor, foi investido R\$ 				880.050,00 (Oitocentos de Oitenta Mil e Cinquenta Reais) pela UECE, através da fonte Custeio e R\$ 876.100,00 					(Oitocentos e Setenta e Seis Mil e Cem Reais) pelo Fundo de	Combate a Pobreza - FECOP. O fundo mencionado, é 					gerenciado pela Fundação Cearense de Apoio ao Desenvolvimento Científico e Tecnológico - FUNCAP, mantida pelo Governo 		do Estado do Ceará.
		
			\begin{figure}[htb]
        		\centering
        		\includegraphics[scale=0.6]{Figura_4.png}
        		\caption{Recursos por Fonte de Pagamento}
        		\label{Monitores-Centro/Faculdade}
    		\end{figure}
    		
    	\pagebreak

		\subsection{Qual a distribuição dos monitores por fonte de pagamento?}
		
		Nesta seção, apresentamos um gráfico que contém a distribuição dos monitores levando em consideração as fontes de 				fomento FECOP e Custeio, E também, a quantidade de alunos voluntários.
		
			
			\begin{figure}[htb]
        		\centering
        		\includegraphics[scale=0.6]{Figure_5.png}
        		\caption{Distribuição de Monitores por Fonte de Pagamento}
        		\label{Monitores por Fonte de Pagamento}
    		\end{figure}
    		
    \section{Conclusão}
    	Embora o período de tempo seja pequeno, os dados não tão abundantes, acreditamos ter atendido as demanda autoimposta 			de garimpar dados do Sistema Bolsas, referentes ao Programa de Monitoria, com o intuíto de entender melhor o que tem 			ocorrido ao longo dos anos.
    	
    	A reflexão mais importante que tiramos dos dados, talvez seja o fato de que o PROMAC, tem crescido ao longo dos 				anos. E isso demanda, por partes de seus gestores, um maior investimento. Basta perceber que na resposta à última 				pergunta, constatamos que maior parte dos monitores foram voluntários. Isso mostra que os alunos da instituição 				anseiam por participar do programa. 
    	
    	Percebemos também que a demanda de investimento vai além do financeiro, acréscimo de bolsas, Mas também em 						treinamento dos funcionários que estão envolvidos diretamente com as suas demandas para que possam continuar 					oferecendo um serviço de qualidade à comunidade Ueceana.
    	
    	

%\bibliographystyle{plain}
\addcontentsline{toc}{section}{Referências}
%\bibliography{referencias.bib}


\end{document}